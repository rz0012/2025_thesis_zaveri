% Chapter Template

\chapter{Conclusions and Future Work} % Main chapter title

\label{Chapter6} % Change X to a consecutive number; for referencing this chapter elsewhere, use \ref{ChapterX}


\section{Conclusions}
We introduce SiamABC, a new Siamese visual tracker that improves the trade-off between the computational requirements and the OOD generalization ability, thus expanding the horizon of applicability of visual trackers in-the-wild under resource constraints. We have shown that it can be as fast as FEAR-XS, while being significantly more accurate with an evaluation over 11 benchmarks. We have also shown the superior ability of SiamABC in OOD generalization by reaching near-SOTA accuracies on the challenging OOD benchmark AVisT, with a significant improvement over the efficient Transformer-based SOTA methods. We credit this achievement to the four major technical contributions of the approach that include the use of a dual-search-region, the fast filtration layer FMT, the TRL loss, and the introduction, for the first time, of the dynamic TTA during tracking. Promising future extensions of this work may include further development of TTA for tracking, and the adoption of tracking inertia.


\section{Future Work}

\section{Potential Negative Societal Impact}
Our goal is mainly to advance the tracking performance of efficient visual tracking in-the-wild. On the other hand, if misused or used with ill-intent, this technology could potentially raise privacy concerns, especially if used for surveillance purposes, or other illegal activities like for stalking, bullying, etc. However, these are general issues that may arise even with other trackers that have been developed in the past. Our hope is to advance technology, and with regulations on its applications, we believe it should be possible to prevent its misuse.